\documentclass[12pt]{article}

\usepackage[utf8]{inputenc}
\usepackage{amsmath}
\usepackage{amssymb}
\usepackage{xcolor}
\usepackage{mathtools}
\newtheorem{theorem}{Theorem}[section]
\newtheorem{corollary}{Corollary}[theorem]
\newtheorem{lemma}[theorem]{Lemma}
\newtheorem{proposition}[theorem]{Proposition}
\DeclarePairedDelimiter{\ceil}{\lceil}{\rceil}
\title{Small ideas}

\begin{document}

\maketitle

\section{complete graphs, small ideas}

Suppose the weights are independent variables $N(0,1)$.\\
A fixed move brings an improvement of $N(0, n)$\\
\begin{equation*}
P(move \in [0,\epsilon]) = \int_0^\epsilon \frac{1}{\sqrt{2\pi n}} exp(\dfrac{-x^2}{2n}) = \frac{1}{2} erf(\frac{\epsilon}{\sqrt{2n}})
\end{equation*}

We can take the Taylor series :
\begin{equation*}
\begin{split}
\frac{1}{2} erf(\frac{\epsilon}{\sqrt{2n}}) &= \dfrac{1}{\pi}\sum_{i = 0}^{\infty}\frac{\epsilon}{(2i + 1)\sqrt{2n}}\prod_{k = 1}^{i}\frac{-\epsilon^2}{2kn}\\
&\leq \dfrac{\pi^{3/2} \epsilon^3}{6n\sqrt{2n}}
\end{split}
\end{equation*}

However we fail to have a good union-bound here.
\end{document} 
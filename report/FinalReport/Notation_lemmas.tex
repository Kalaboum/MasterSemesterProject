\section{Notation and preliminary lemmas}

Let $X = (X_e)_{e \in E} \in [-1, 1]^E$ a random vector with independent entries, corresponding to the edge weights. We assume that $X_e$ has density $f_e$ with respect to the Lebesgue mesasure, and we denote $\phi = \max_{e \in E}||f_e||_\infty$. 

\begin{lemma} (Lemma 2.1 \cite{angel2016local})\\
\label{noise}
Let $\gamma_1, ..., \gamma_k$ be $k$ linearly independent vectors in $\mathbb{Z}^E$. Then the joint density of $(\langle \gamma_i, X \rangle)_{i \leq k}$ is bounded by $\phi^k$. In particular, if sets $J_i \in \mathbb R$ have measure at most $\epsilon$ each, then 
\begin{equation*}
\mathcal{P} (\forall i \in [k], \langle \gamma_i, X \rangle \in J_i) \leq (\phi \epsilon)^k
\end{equation*}
\end{lemma}

We do not give here a proof for this lemma, see the reference \cite{angel2016local} for a detailed proof.\\
Define a move vector $\alpha(v)$ as a vector indexed by E whose entries are :
\begin{equation*}
\alpha(v)_{uw} = 
\begin{cases}
\sigma(u)\sigma(w) &\text{ if } uw \in E \text{ and }( (u = v) \text{ or } (w = v)) \\
 0 &\text{ otherwise}
 \end{cases}
\end{equation*} 

For a sequence $L = (v_1, ..., v_l)$ a word of vertices and initial state $\sigma_0$, let $\alpha(v_i), i \in [l]$ be the corresponding move vectors. Let $\sigma_t$ be the state just after flip of vertex $v_t$, remark that it depends only on the initial state $\sigma_0$ and the vertices in the sequence.
The improvement made by a move is defined as :
\begin{equation*}
h(\sigma_t) - h(\sigma_{t-1}) = \langle \alpha(v_t), X \rangle
\end{equation*}
Where $X$ is the edge weight vector.\\ 
A matrix $A_L$ is defined as the concatenation of the move vectors as columns. We call a sequence $\epsilon$-slowly improving if all moves yield an improvement of at most $\epsilon$.

We introduce here the concept of critical block. A block $B$ is defined as a substring of a sequence $L$.

Let $S(L)$ be the set containing the distinct vertices in $L$ and $s(L)$ be its cardinality, $s_1(L)$ the number of distinct vertices in $L$ that appear only once, $s_2(L)$ the number of distinct vertices in $L$ that appear multiple times. Let $l(B)$ be the length of the block, i.e. the number of moves.
A block $B$ is critical if: for $\beta > 0$, $l(B) \geq (1 + \beta)s(B)$ and every block $B'$ strictly contained in $B$ has $l(B') < (1+\beta)s(B')$.

\begin{lemma}
\label{critical}
(Lemma 4.1 \cite{angel2016local}) \\
For complete graphs with n vertices: fix any positive integer $n \geq 2$ and a constant $\beta > 0$. Given a sequence consisting of $s(L) < n$ letters and with length $l(L) \geq (1 +\beta)s$, there exists a critical block $B$ in $L$. Moreover, a critical block satisfies $l(B) = \ceil{(1 + \beta)s(B)}$.
\text{Moreover }$\rank(B) \geq \dfrac{1+4\beta}{1 + 3\beta}s(B)$
\end{lemma} 

This is the result on which \cite{angel2016local} based their proof for the complexity of complete graph.

\begin{lemma} \leavevmode \\
\label{lem::boundN}
Given $J_1, ... J_l$ sets composed of at most k closed intervals each, with $J_t$ having at most measure $\epsilon$.
If $l(L) = l, s(L) < n$, and $\rank(L) \geq s(L)$ then \\
\begin{equation*}
P(\langle \alpha(v_t), X \rangle \in J_t \quad \forall t\leq l)  \leq (2n/\epsilon)^{s_0}(16(2k+1)\phi\epsilon)^{\rank(A_L)}
\end{equation*}
Where $s_0$ is the number of vertices that have at least one neighbour which is not in L. 
\end{lemma}

\textit{Proof.} Let $I$ be the set of the edges corresponding to independent rows in $A_L$. They do not depend on $\sigma_0$ since the starting configuration only multiplies each row by 1 or -1.\\
Define $T = \{v \in V: vw \in I \text{ or } v \in S(L)$\}. $|T| \leq 2 \rank(A_L) + s(L)$. \\
We split $h(\sigma)$ in three part $h_0(\sigma), h_1(\sigma), h_2(\sigma)$ Where $h_i$ is the restriction of $h$ to the edges that have $i$ endpoints in $T$. \\
$h_0(\sigma_t) - h_0(\sigma_{t -1}) = 0$. Since the edges whose both endpoints are not flipped do not provoke a change in the total weight cut. \\
\begin{equation*}
\begin{split}
&h_1(\sigma_t) - h_1(\sigma_{t-1}) = -\sigma_t(v_t)\sum_{u \not\in T, uv_t \in E}\sigma_0(u)X_{uv_t} \\
=& \sigma_t(v_t)Q(v_t).
\end{split}
\end{equation*}
where $(Q(v_t) = -\sum_{u \not\in T, uv_t \in E}\sigma_0(u)X_{uv_t})$. Since $|X_e| \leq n$ and the maximum neighbours of a vertex is n, $Q(v_t) \in [-n, n]$. By defining $D = 2\epsilon\mathbb Z \cap [-n, n]$, there exists some $d(v_t) \in D$ such that $|Q(v_t) - d(v_t)| \leq \epsilon$.\\
\begin{equation*}
\begin{split}
h_2(\sigma_t) - h_2(\sigma_{t-1}) = &\langle\alpha ', X \rangle \text{ where }\\
\alpha_t ' =  &
\begin{cases}
-\sigma_t(u)\sigma_t(w) & uw \in E, v_t \in \{u, w\}, \{u, w\} \subseteq T \\
0 & \text{otherwise}
\end{cases}
\end{split}
\end{equation*}
Since $\alpha_t'$ concerns only the rows linearly independant from $A_L$, $\rank([\alpha_t']_{t \leq l(L)}) = \rank(A_L)$. \\
\begin{equation*}
h(\sigma_t) - h(\sigma_{t-1}) = \langle\alpha ', X \rangle +   \sigma_t(v_t)d_t + \delta_t \text{    where }|\delta_t| \leq \epsilon
\end{equation*}

Since  $|\delta_t| \leq \epsilon$, and $h(\sigma_t) - h(\sigma_{t-1}) \in J_t$ : 
\begin{equation*}
\exists j \in J_t, |j - \langle\alpha ', X \rangle -   \sigma_t(v_t)d_t| \leq \epsilon
\end{equation*}
So the term $\langle\alpha ', X \rangle +   \sigma_t(v_t)d_t$ lies either in one of the k closed intervals $J^k_t$ that form $J_t$ either in an interval of measure $\epsilon$ just after, or just before $J^k_t$. This whole new set S is of measure at most $(2k+ 1)\epsilon$. \\
As $\sigma_t(v_t)$ is -1 or 1, $\langle\alpha ', X \rangle$ lies in a set defined as :
\begin{equation*}
S_{final} = \{p : p - d_t \in S\} \cup \{q:  q + d_t \in S\}
\end{equation*}
This set is of measure at most $2(2k + 1) \epsilon$. Using lemma \ref{noise}, the probability that $\langle\alpha ', X \rangle$ belong to $S_{final}$ is at most $2(2k + 1)\phi\epsilon^{\rank(A_L)}$. Using union bound over $\sigma_{t \in T}$ and $d$ :
\begin{equation*}
P(\langle \alpha(v_t), X \rangle \in J_t) \quad \forall t\leq l) \leq 2^{2\rank(A_L) + s}(\frac{2n}{\epsilon})^{s_0}(2(2k+1))\phi\epsilon)^{\rank(A_L)}
\end{equation*}
Remark the $s_0$ instead of $s$ in exponent, since vertices that have no-non flipped neighbors need not be taken in this bound. \\
By using  $\rank(L) \geq s(L)$, we get the desired bound. \\

\begin{corollary}
\label{cor::boundN}
If $l(L) = l, s(L) < n$, and $\rank(L) \geq s(L)$ then \\
P(L is $\epsilon$-slowly improving from some state $\sigma_0 \leq (2n/\epsilon)^{s_0}(48\phi\epsilon)^{\rank(A_L)}$
\end{corollary}

It follows from the previous lemma, by replacing all $J_t$ by $[0, \epsilon]$.
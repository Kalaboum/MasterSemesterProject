\section{Proof for graphs with multiple distant cliques}
Let $G = (V,E)$ be a graph with $n$ vertices and $m$ edges. Assume that this graph contain a family $H$ of cliques $H_1, ... H_h$ of $r_1, ... , r_h$ vertices and that the degree of vertices in the set $G \setminus \bigcup_{i = 1}^h H_i$ is at most $log(n)$. Furthermore, forall $1 \leq i < j \leq h$ and any pair of two vertices $v$ and $w$ such that  $v \in H_i, w \in H_j$; we have $v \neq w, vw \not \in E$\\
Any pair of cliques is distant, meaning that a path going from a vertex of one clique to the other is of length at least 2. \\
The precedent proof can be easily extended to those graphs. We will prove a similar proposition, but on longer sequences.

\begin{proposition}
\leavevmode \\
With high probability, there exists no $\epsilon$-slowly improving sequence of length $2n^2$ from any starting configuration $\sigma_0$, for $\epsilon$ is O(1/poly(n)).
\end{proposition}

\begin{lemma}
\label{edgeDisjoint}
Let $L$ be a sequence of q moves such that $S(L) \subseteq \bigcup_{i \leq k}A_i  \subseteq V$, where $A_1, ... , A_k$ are edge-disjoint sets. Then, there exists a sequence with the same vertices but a different ordering on the moves such that $\forall l < j \leq q, \text{ if } v_l \in A_i \text{ and } v_j \in A_i, \text{then } v_d \in A_i \quad\forall l \leq d  \leq j$   
\end{lemma}
\textit{Proof.} The proof is very straightforward. Suppose we have $v_t$ and $v_{t+1}$ which are distant, the amelioration brought by $v_t$ is equal to :
\begin{equation*}
-\sigma_t(v_t) \sum_{u \in V, uv_t \in E}w_{uv_t}\sigma_t(u) = -\sigma_{t+1}(v_t) \sum_{u \in V, uv_t \in E}w_{uv_t}\sigma_{t+1}(u_t) 
\end{equation*}
We recall that $\sigma_t$ is the partition just after the flip of $v_t$. The equality above holds because $\sigma_t$ and $\sigma_{t+1}$ differ only by the value of $v_{t+1}$ and $ (v_t v_{t+1})$ does not belong to $E$. This means that we can flip $v_t$ at time $t$ or $t+1$, this is true also for $v_{t+1}$. We can then swap them, and both are still improving moves with the same amelioration of total weight. By repeating the swaps, we reach a new sequence where all moves of vertices $\in A_i \quad \forall i \leq k$ are consecutive.\\

If we consider now sequences of length $2n^2$. Either there is a vertex with logarithmic degree and the whole sequence is not $\epsilon$-slowly improving either we can reorder them with the previous lemma, such that vertices belonging to the same clique are consecutive. Since the number of cliques is upperbounded by $n$, by pigeonhole argument we have at least one sequence of size at least 2$n$, which contains vertices from only a clique. We showed that with high probability such a sequence is not $\epsilon$-slowly improving. The whole sequence is then not $\epsilon$-slowly improving, concluding the proof.
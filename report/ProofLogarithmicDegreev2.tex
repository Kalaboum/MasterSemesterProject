

\documentclass[12pt]{article}

\usepackage[utf8]{inputenc}
\usepackage{amsmath}
\usepackage{xcolor}
\newtheorem{theorem}{Theorem}[section]
\newtheorem{corollary}{Corollary}[theorem]
\newtheorem{lemma}[theorem]{Lemma}

\title{Smoothed Complexity of Local Search of Max-Cut for graph with logarithmic degree}
\begin{document}


\maketitle

\section{Introduction}

Let $G(V,E)$ be a graph with a set $V$ of n vertices and $E$ of m edges. Forall vertex v that belongs to $V$, the degree of v is inferior or equal to $c \cdot log(n)$ where c is a constant. Let $ w : E \rightarrow ]-1, 1[$ be an edge weight function. This restriction is for commodity in the proof, we can easily generalize to other weight function by dividing all weight by $w_{max} = max_e (|w_e|)$ as long as the variance of the weight function is bounded. \\
%TODO ma<be use a slightly different notation
Let $\sigma : V \rightarrow {-1,1}$ be a partition, $\sigma_0 a starting configuration$ and let the size of the cut corresponding to this partition be :
\begin{equation*}
\dfrac{1}{2}\sum_{uv \in E} {w(uv)(1 - \sigma(u)\sigma(v)) }
\end{equation*}

We call a cut locally optimal, if there exists no vertex v such that if v is moved to the other side of the cut, the size of the cut increases. 
A simple algorithm called FLIP solves the problem of finding a local-optimal cut. Search for a vertex which, if flipped, improves the size of the cut (we call it an improving move), if there is none stop, else repeat. Let call $s$ such a sequence of move \\

Formally we now assume that the weight on edge $e \in E$ is given by a random variable $X_e \in [-1,1]$ which has a density with respect to the Lebesgue measure bounded from above by $\phi$ (for example this forbids $X_e$ to be too close to a point mass). We assume that these random variables are independent.\\
We will prove here that the smoothed complexity is polynomial. 

%TODO deal with the noise

\section{Proof}
\subsection{Bounding a move}

%TODO adapt this
Let call $t$ a move. Remark that $\alpha_t$, the amelioration to the cut provided by the move t, depends only on the vertex which is flipped and its neighbors. 
Here we use a particular case of the lemma 7 provided by Etscheid and Röglin[2014]. \\ 
\begin{lemma}
For $\epsilon > 0$ and a fixed move $t$ $\mathbf{Pr}(\alpha_t \in [0,\epsilon]) \leq \phi \epsilon$ \\
\end{lemma}

\subsection{Bounding all moves using union bound}

We use the union bound over the n possible vertices and the $2^{c \cdot log(n)}$ possible starting configuration. This gives us that for any move : \\
 $\mathbf{Pr}(\exists \sigma_0$ : an improvement made by a move $\in [0,\epsilon]) \leq n2^{clog(n)}\phi \epsilon$ \\
 
 \subsection{Finding a good epsilon}
 Now we choose $\epsilon = {n^{-(1 + c + \eta)} \phi^{-1}}$ with $\eta > 0 $. Thus, with high probability, each move improves the cut by at least $\epsilon$. 
 
 Since the cut is in $]-cnlog(n), cnlog(n)[$. The number of steps is at most $\dfrac{2cnlog(n)}{\epsilon}$. 

The complexity is polynomial with $O(n^{2 + c + \eta}log(n)\phi)$\\

\end{document} 
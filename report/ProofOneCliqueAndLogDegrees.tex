\documentclass[12pt]{article}

\usepackage[utf8]{inputenc}
\usepackage{amsmath}
\usepackage{amssymb}
\usepackage{xcolor}
\usepackage{mathtools}
\newtheorem{theorem}{Theorem}[section]
\newtheorem{corollary}{Corollary}[theorem]
\newtheorem{lemma}[theorem]{Lemma}
\newtheorem{proposition}[theorem]{Proposition}
\DeclarePairedDelimiter{\ceil}{\lceil}{\rceil}
\title{Smoothed Complexity of Local Search of Max-Cut for graph with one clique and other  vertices have logarithmic degree}

\begin{document}

\maketitle

\section{Introduction}

Let $G = (V,E)$ be a graph with $n$ vertices and $m$ edges. Assume that this graph contains a clique $H$ of $r$ vertices and that the degree of vertices in the set $G \setminus H$ is at most $log(n)$. \\
Let $w : E \rightarrow [-1, 1]$ be an edge weight function. The local max-cut problem consists in finding a partition of the vertices $\sigma$ such that the total cut weight, defined as :
\begin{equation*}
h(\sigma) = \dfrac{1}{2}\sum_{uv \in E}(1 - \sigma(u)\sigma(v))w(uv)
\end{equation*}
is locally optimal. By locally optimal, we mean that there exists no vertex v such that, if we flip the vertex, i.e change the sign of $\sigma(v)$, then $h(\sigma)$ increases.\\ 

A really naive algorithm called FLIP solves this problem. It finds a vertex which if flipped leads to an amelioration of the total weight cut, flips it and repeat until no such vertex exists. Some instances of graphs have been found to have an exponential number of steps before terminating. However, for most of the graphs, FLIP terminates in a reasonable time. Moreover, when adding a small amount of noise to the complicated graphs, FLIP's running time improved greatly. This lead to the study of the smoothed complexity of FLIP, in which noise is added to the edge weights.

\section{Notation and preliminary lemmas}

Let $X = (X_e)_{e \in E)} \in [-1, 1]^E$ a random vector with independent entries, corresponding to the edge weights. We assume that $X_e$ has density $f_e$ with respect to the Lebesgue mesasure, and we denote $\phi = max_{e \in E}||f_e||_\infty$. 

\begin{lemma} (Lemma 2.1 [3])\\
\label{noise}
Let $\alpha_1, ..., \alpha_k$ be $k$ linearly independent vectors in $\mathbb{Z}^E$. Then the joint density of$ (\langle \alpha_i, X \rangle)_{i \leq k}$ is bounded by $\phi^k$. In particular, if sets $J_i \in \mathbb R$ have measure at most $\epsilon$ each, then 
\begin{equation*}
\mathcal{P} (\forall i \in [k], \langle \alpha_i, X \rangle \in J_i) \leq (\phi \epsilon)^k
\end{equation*}
\end{lemma}

This lemma will prove determinant for the proof. This is the only part where we need the noise.

We define a move vector $\alpha_v$ as a vector indexed by E whose entries are :
\begin{equation*}
\alpha_{uw} = \sigma(u)\sigma(w) \text{ if } uw \in E \text{ and }( (u = v) \text{ or } (w = v)), 0 \text{ otherwise}
\end{equation*} 

For a sequence $L = (v_1, ..., v_l)$ of l moves and initial state $\sigma_0$, let $\alpha_i, i \in [l]$ be the corresponding move vectors. Let $\sigma_t$ be the state just after flip of vertex $v_t$. We define matrix $A_L$ has the concatenation of the move vectors as columns.

We call a sequence $\epsilon$-slowly improving if all moves yield an improvement of at most $\epsilon$.

\begin{proposition}
\label{prop}
With high probability, there exists no $\epsilon$-slowly improving sequence of length $2n$ from any starting configuration $\sigma_0$, for $\epsilon$ is O(1/poly(n)).
\end{proposition}

Proving this proposition implies that the smoothed complexity is O(poly(n)). \\
If there exists no such sequence, then $2n^2/\epsilon$ sequence of $2n$ moves yield an improvement of at least $2n^2$ which is the maximum improvement possible since $h(\sigma) \in [-n^2,n^2]$. \\
Thus number of steps is $O(n^3/\epsilon)$ which is $O(poly(n))$.

We introduce here the concept of critical block. A block B is defined as a subsequence of a sequence L.\\
Let $S(L)$ be the set containing the distinct vertices in L and s(l) be its cardinality, $s_1(L)$ the number of distinct vertices in L that appear only once , $s_2(L)$ the number of distinct vertices in L that appear multiple times. Let $l(B)$ be the length of the block, i.e the number of moves.
A block B is critical if $l(B) \geq (1 + \beta)s(B)$ and every block $B'$ strictly contained in $B$ has $l(B') < (1+\beta)s(B')$.

\begin{lemma}
\label{critical}
(Lemma 4.1 [3]) \\
For complete graphs with n vertices: fix any positive integer $n \geq 2$ and a constant $\beta > 0$. Given a sequence consisting of $s(L) < n$ letters and with length $l(L) \geq (1 +\beta)s$, there exists a critical block $B$ in $L$. Moreover, a critical block satisfies $l(B) = \ceil{(1 + \beta)s(B)}$.
\text{Moreover }$rank(B) \geq \dfrac{1+4\beta}{1 + 3\beta}s(B)$
\end{lemma} 

\begin{lemma}(lemma 4.4 [3])\\
\label{bound}
P(L is $\epsilon$-slowly improving from some $\sigma$) $\leq 2 (\dfrac{4n}{\epsilon})^s (8\phi \epsilon)^{rank(A_L)}$

\end{lemma}


\section{Proof of Proposition \ref{prop}}

\begin{lemma}
If $s(L) < n$, then \\
\begin{equation*}
P(L \text{ is }\epsilon\text{-slowly improving from some }sigma_0)  \leq (2n/\epsilon)^{s_0}(64\phi\epsilon)^{rank(A_L)}
\end{equation*}
Where $s_0$ is the number of vertices that have at least one neighbor which is not in L. 
\end{lemma}

\textit{Proof.} Let $I$ be the set of the edges corresponding to independent rows in $A_L$. They do not depend on $\sigma_0$ since the starting configuration only multiplies each row by 1 or -1.\\
Define $T = \{v \in V: vw \in I \text{ or } v \in S(L)$\}. $|T| \leq 2 rank(A_L) + s(L)$. 
We split $h(\sigma)$ in three part $h_0(\sigma), h_1(\sigma), h_2(\sigma)$ Where $h_i$ is the restriction of $h$ to the edges that have $i$ endpoints in $T$. \\
$h_0(\sigma_t) - h_0(\sigma_{t -1}) = 0$. Since the edges whose both endpoints are not flipped do not provoke a change in the total weight cut. \\
\begin{equation*}
\begin{split}
h_1(\sigma_t) - h_1(\sigma_{t-1}) = -\sigma_t(v_t)\sum_{u \not\in T, uv_t \in E}\sigma_0(u)X_{uv_t} \\
= \sigma_t(v_t)Q(v_t).
\end{split}
\end{equation*}
%%TODO pursue here !


We will prove the proposition by considering different sequences of size 2n.\\
$E_1$ is the event corresponding to $\exists L, \sigma_0, \text{ s.t. L is }\epsilon\text{-slowly improving from }\sigma_0 \text{ and }S(L) \not\subseteq H$.\\
$E_2$ is the event corresponding to $\exists L, \sigma_0,\text{ s.t. L is }\epsilon\text{-slowly improving from }\sigma_0\text{ and }S(L) \subseteq H and s(L) < r$. \\
$E_3$   is the event corresponding to $\exists L, \sigma_0,\text{ s.t. L is }\epsilon\text{-slowly improving from }\sigma_0\text{ and }S(L) \subseteq H and s(L) = r$ \\

Consider $E_1$. Fix $L$ and $\sigma_0$ then there must be a vertex $v \in S(L)$ whose degree is at most $log(n)$. By lemma  \ref{noise}, the probability that $\alpha_v \in [0, \epsilon] = \phi \epsilon$. Using union bound on the number of vertices and the starting configuration of those vertices we have:
\begin{equation}
P(E_1)  \leq 2^{log(n)}n \phi \epsilon
\end{equation}
By taking$\epsilon = n^{-2 - \eta} \phi^{-1}$ with$\eta > 0$ there exists no such sequence with high probability. \\

Now consider $E_2$. We consider the subgraph G' induced by the restriction to the clique H. We observe that $rank_G'(A_L) = rank_G(A_L)$ Since each row corresponding to the edges that has one or zero endpoints in H can be expressed has a linear combination of the rows corresponding to the edges that have two endpoints in H. \textcolor{red}{TODO prove it}. \\
By lemma \ref{critical}, there exists a critical block B whose rank is at least 1.25 s(B).We now can use lemma \ref{bound} to have this bound:
\begin{equation*}
P(B \text{ is }\epsilon \text {-slowly improving from some }\sigma) \geq 2(\frac{4n}{\epsilon})^{s(B)}(8\phi\epsilon)^{\frac{5s(B)}{4}}
\end{equation*}
Because the number of blocks using s letters is $n^{2s}$, we have:
\begin{equation*}
P(E_2) \geq 2 \sum_{s < n}(64\phi^{5/4}n^3\epsilon^{1/4})^s
\end{equation*}
By choosing $\epsilon = n^{-(12 + \eta)}\phi^{-5}$ This sums goes to zero.


\end{document} 